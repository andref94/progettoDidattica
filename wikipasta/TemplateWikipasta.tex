%%%%%%%%%%%%%%%%%%%%%%%%%%%%%%%%%%%%%%%%%%%%%%%%%%%%%%%%%%%%%%%%%%%%%%%%%%%
%                                                                         %
%		               SCHEDA LABORATORIO ALADDIN                         %
%			             ______________________                           %
%                                                                         %
%            			 AUTORE: Andrea Formica                           %
%                                                                         %
%			           Ultima revisione: 29.06.2018                       %
%                                                                         %
%%%%%%%%%%%%%%%%%%%%%%%%%%%%%%%%%%%%%%%%%%%%%%%%%%%%%%%%%%%%%%%%%%%%%%%%%%%
%
%
\documentclass[12pt]{article}
%
\usepackage[utf8]{inputenc}
\usepackage[english]{babel}
 
\setlength{\parindent}{2em}
\setlength{\parskip}{1em}

% per le accentate
\usepackage[utf8]{inputenc}
%
%
%			TITOLO
\title{Wikipasta}
\author{A. Formica, G. Garbin, S. Patania}
%
\begin{document}
\maketitle
%
% 
\section{Temi}
Introduzione alla formattazione di un testo tramite un linguaggio di markup.
%
%
\section{Descrizione}
I ragazzi vengono divisi a coppie: ad ogni coppia viene distribuito un testo non formattato, lo stesso testo formattato e degli oggetti. Ai ragazzi viene chiesto di definire, utilizzando liberamente gli oggetti messi a disposizione, delle regole di codifica per la formattazione proposta e di applicarle al testo non formattato. In un secondo momento vengono introdotti dei vincoli da rispettare nell’elaborazione della propria codifica. Infine ogni coppia, attraverso un apposito software, deve formattare alcuni testi per sperimentare ciò che è stato appreso nella fase precedente.
%
%
\section{Target}
Questo laboratorio è rivolto agli studenti degli ultimi due anni della scuola primaria e della scuola secondaria di primo grado.
%
%
\section{Obiettivi formativi}
\subsection{Conoscenze}
\begin{itemize}
\item sapere come fanno i computer a formattare un testo;
\item conoscenza del linguaggio tipo wiki proposto in classe;
\item concetto di marcatore;
\end{itemize}

\subsection{Abilità}
\begin{itemize}
\item definire regole per la formattazione di un testo;
\item elaborare una codifica per un testo formattato;
\item interpretare un testo formattato
\item saper stabilire se due rappresentazioni sono equivalenti;
\end{itemize}

\subsection{Competenze}
\begin{itemize}
\item rendersi conto che lo stesso testo formattato si può rappresentare in modi differenti;
\item risolvere un problema rispettando determinati vincoli;
\item interpretare e utilizzare le regole sintattiche di un qualunque linguaggio di markup;
\end{itemize}
%
%
\section{Spazi, tempi e materiali}
Il laboratorio non ha bisogno di particolari spazi. Bisogna però tenere presente che l'ultima attività si svolge a coppie al computer.

La durata complessiva del laboratorio (inclusi l'introduzione e il riepilogo conclusivo) è di 2 ore.

Nello svolgersi dell'attività si alternano attività a classe intera con attività svolte in piccoli gruppi.

Per realizzare il laboratorio occorrono i seguenti materiali:
\begin{itemize}
\item testo semplice
\item testo formattato
\item oggetti vari per formattare il testo (pasta, perline, elastici, legumi)
\item software Wikipasta
\end{itemize}
%
%
\section{Parole chiave}
Linguaggio di markup, codifica dell'informazione
%
% 
\end{document}


 
