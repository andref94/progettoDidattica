%%%%%%%%%%%%%%%%%%%%%%%%%%%%%%%%%%%%%%%%%%%%%%%%%%%%%%%%%%%%%%%%%%%%%%%%%%%
%                                                                         
%		               SCHEDA LABORATORIO ALADDIN                         
%			             ______________________                           
%                                                                         
%            			 AUTORE: Andrea Formica                           
%                                                                         
%			           Ultima revisione: 29.05.2018                       
%                                                                         
%%%%%%%%%%%%%%%%%%%%%%%%%%%%%%%%%%%%%%%%%%%%%%%%%%%%%%%%%%%%%%%%%%%%%%%%%%%
%
%
\documentclass[12pt]{article}
%
\usepackage[utf8]{inputenc}
\usepackage[english]{babel}
 
\setlength{\parindent}{2em}
\setlength{\parskip}{1em}

% per le accentate
\usepackage[utf8]{inputenc}
%
%
%			TITOLO
\title{Labirinti}
\author{A. Formica, G. Garbin, S. Patania}
%
\begin{document}
\maketitle
%
% 
\section{Temi}
Introduzione alla programmazione attraverso un laboratorio attivo (pratico) e all’utilizzo di un linguaggio di programmazione visuale (Scratch).
%
%
\section{Descrizione}
Il laboratorio viene introdotto spiegando il significato del termine informatica, ovvero elaborazione automatica dell’informazione e viene chiesta ai ragazzi la loro definizione di automatico così da ottenerne una condivisa da tutta la classe. Successivamente i ragazzi vengono suddivisi in gruppi di 3/4 persone: un ragazzo impersona un “robot” che deve essere guidato dai compagni attraverso un percorso prestabilito. In un primo momento i gruppi prendono dimestichezza con il percorso e le istruzioni, poi gli vengono posti dei vincoli da rispettare nell'elaborazione della soluzione. Al termine del laboratorio i ragazzi lavoreranno a coppie al computer con Scratch per la risoluzione di alcuni labirinti.
%
%
\section{Target}
Questo laboratorio è rivolto agli studenti degli ultimi due anni della scuola primaria, della scuola secondaria di primo grado e dei primi due anni della scuola secondaria di secondo grado. \`E prevista una piccola diversificazione delle parti che compongono il laboratorio a seconda dell'età degli studenti..
%
%
\section{Obiettivi formativi}
\subsection{Conoscenze}
\begin{itemize}
\item Concetto di algoritmo come sequenza finita di istruzioni, non ambigue, da seguire “ciecamente” senza interruzioni dalla prima all’ultima
\item Principali strutture di controllo (costrutti condizionali ed iterativi) supportate dai linguaggi di programmazione
\end{itemize}

\subsection{Abilità}
\begin{itemize}
\item Saper individuare le operazioni primitive da utilizzare nell’algoritmo
\item Essere in grado di comporre istruzioni primitive e strutture di controllo per realizzare comandi più complessi
\item Saper utilizzare un software (Scratch) per la programmazione visuale
\end{itemize}

\subsection{Competenze}
\begin{itemize}
\item Individuare le strategie che riducono il numero di operazioni primitive di un algoritmo
\item Imparare a discriminare differenti strategie risolutive in termini di generalità
\item Riconoscere i costrutti iterativi nello schema risolutivo di un dato problema
\item Affinare alcune proprietà del pensiero computazionale quali problem solving, lavoro di gruppo e astrazione della strategia risolutiva
\end{itemize}
%
%
\section{Spazi, tempi e materiali}
Il laboratorio si può svolgere in una comune aula scolastica, purchè si possano spostare i banchi per la realizzazione del percorso che verrà fatto fare ai robot. Inoltre è necessaria un'aula informatizzata per l'ultima parte del laboratorio (per evitare perdite di tempo sarebbe opportuno utilizzare lo stesso spazio per la prima e la seconda parte).

La durata complessiva del laboratorio (inclusi l'introduzione e il riepilogo conclusivo) è di 2 ore.

Nello svolgersi dell'attività si alternano attività a classe intera con attività svolte in piccoli gruppi.

Per realizzare il laboratorio occorrono i seguenti materiali:
\begin{itemize}
\item post-it di quattro colori differenti
\item carta e penna
\item bende per gli occhi (una per ogni robot)
\item software Scratch e file esercizi
\end{itemize}
%
%
\section{Parole chiave}
Linguaggi di Programmazione Visuale, Strutture di Controllo, Algoritmo
%
% 
\end{document}


 
