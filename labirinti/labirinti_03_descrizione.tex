%%%%%%%%%%%%%%%%%%%%%%%%%%%%%%%%%%%
%                                                                         
%		                 Labirinti: descrizione attività                         
%			                 ______________________                           
%                                                                         
%            			 AUTORE: Sabrina Patania                           
%                                                                         
%			            Ultima revisione: 08.06.2018                        
%                                                                        
%%%%%%%%%%%%%%%%%%%%%%%%%%%%%%%%%%%
%
%
\documentclass[12pt]{article}
%
\usepackage[utf8]{inputenc}
\usepackage[english]{babel}
 
\setlength{\parindent}{2em}
\setlength{\parskip}{0em}

% per le accentate
\usepackage[utf8]{inputenc}
%
%
%			TITOLO
\title{Labirinti: descrizione}
\author{A. Formica, G. Garbin, S. Patania}
%
\begin{document}
\maketitle
%
% 
\section{Presentazione}
Il laboratorio inizia con una veloce presentazione delle varie attività.
In questa prima fase è importante rassicurare gli studenti che non esistono risposte giuste o sbagliate, ogni idea/proposta è benvenuta, purché gli studenti siano disponibili a metterle in discussione e ragionarci sopra.

Agli studenti viene chiesto di riflettere sulla parola ?informatica?, spronandoli a dare una loro definizione partendo dalle loro intuizioni. L'insegnante raccoglie gli spunti degli studenti per costruire la definizione di informatica come ?elaborazione automatica dell?informazione?. Dopodiché segue una breve descrizione dei due termini che non riguardano il laboratorio, in questo caso ?elaborazione? e ?informazione?.

\begin{enumerate}
\item \textbf{elaborazione}. Un?informazione può essere usata per ottenerne un?altra, ad esempio  l?informazione ?Ada va a scuola?, può essere trasformata in Ada è una studentessa, esistono studenti che portano il nome di Ada, eccetera ?
\item \textbf{informazione}. Ciò che sappiamo o che osserviamo può essere rappresentato, cioè distinto in modo preciso dal resto dell?universo. Risultano particolarmente interessanti le rappresentazioni simboliche (digitali/alfabetiche), che con un ristretto numero di simboli (alfabeto) rappresentano oggetti/concetti. Un linguaggio con una ventina di simboli rappresenta centinaia di migliaia di parole, le quali possono essere composte per rappresentare concetti concreti e astratti.

Etimologia della parola informatica = informazione + matematica.

\textbf{Suggerimenti e cose alle quali fare attenzione}.
È importante rivolgersi alla classe, muovete lo sguardo, cercate di mostrarvi tranquilli e aperti, per metterli a loro agio.

\end{enumerate}
\section{Verso una definizione di automatico}
Nella prima attività del laboratorio viene consegnato a ciascun studente un post-it e viene chiesto loro di scrivere una frase (costituita da una decina di parole) che definisca cosa vuol dire ?automatico?. Questi post-it devono rimanere anonimi.
Si leggono a voce alta i post-it, cercando di rivolgersi alla classe, e man mano si raccolgono e si clusterizzano sul cartellone in base ai concetti chiave che emergono dalle definizioni che hanno dato gli studenti.
Alla fine si provano a riepilogare questi concetti chiave, e si sottolinea come dai diversi contributi emerga una definizione varia e ricca della parola ``automatico".
Il senso di questa attività è costruire una definizione collettiva di cui la classe è autrice; è importante che ciascuno riconosca e senta valorizzato il suo contributo, quindi bisogna fare attenzione a:
\begin{enumerate}
\item leggere le frasi con calma, dando il tempo a tutti di capire il contributo dei compagni
\item durante la lettura dei post-it, se qualcuno si fa riconoscere come autore, darne un riscontro sorridendo o con una conferma verbale
\item usare come parole chiave i termini o le espressioni usate davvero nei post-it e non dei sinonimi (ad esempio ?autonomo? e ?da solo? non sono la stessa espressione)
\item non forzare le associazioni/raggruppamenti di post-it
\item non interpretare i post-it ed evitare di commentarli, ma limitarsi a leggerli
\item è bene incentivare osservazioni o domande relative alla clusterizzazione
\end{enumerate}

Automazione: l?elaborazione può avvenire usando macchine, tutto sommato piuttosto semplici, senza la necessità di intervento umano dopo la progettazione o configurazione iniziale
\section{Robot umani}
Nella seconda attività del laboratorio viene svolto un esercizio di algomotricità.

Gli studenti vengono divisi in 3/4 gruppi (a seconda dello spazio disponibile) e si consegna a ciascun gruppo una benda. Viene predisposto un percorso a L per ragazzini di 8-13 anni e un percorso a S per ragazzi più grandi. In entrambi i casi lungo il percorso si deve posizionare un oggetto che deve essere raccolto durante l'attività.

Ad ogni membro del gruppo viene assegnato un ruolo:

\begin{enumerate}
\item \textbf{robot}: lo studente che interpreta il robot viene bendato e deve seguire in modo acritico le istruzioni impartite dal portavoce
\item \textbf{badante}: segue il robot durante il percorso ed evita che si faccia male
\item \textbf{portavoce}: dando le spalle al robot, dovrà leggere ad alta voce le azioni che il robot dovrà compiere
\end{enumerate}

L'obiettivo del gruppo è condurre un robot umano (un compagno bendato) affinché attraversi tutto il percorso.\\


\textbf{Cose a cui prestare attenzione}.
\begin{enumerate}
\item assicurarsi che la consegna sia chiara, invitando gli studenti a fare domande e chiarire gli eventuali dubbi
\item usare il termine ``azioni" piuttosto che ``istruzioni"
\item questa fase deve durare solo cinque minuti, l'obiettivo è far esplorare e sperimentare il percorso, senza troppi vincoli o regole
\item in questa fase non ha senso aspettarsi che gli studenti inizino già a cogliere le criticità
\end{enumerate}

Nella prima fase della attività gli studenti vengono lasciati liberi di prendere confidenza col percorso e di fare delle prove.

\textbf{prestare attenzione a / suggerimenti}
\begin{enumerate}
\item monitorare tutti i gruppi e sostenere in particolare quelli in difficoltà, quando la maggior parte dei gruppi ha individuato le 4 azioni da scrivere sui post-it, si può passare alla fase successiva
\item la discussione in gruppo dovrebbe favorire la formulazione di idee/domande/critiche: si ha la necessità di spiegare al compagno le proprie idee e perché queste funzionano
\item in questa fase gli studenti effettueranno una vera e propria sperimentazione: se si accorgono che le azioni scelte, o l’ordine con cui sono stati posizionati sul foglio, non consentono al robot di percorrere correttamente il labirinto, inizieranno a domandarsi il perché, e a fare ipotesi su come possono migliorare la loro soluzione
\item non è necessario introdurre in questa fase il termine “programma”, lo si può fare nella fase al computer
\item evitare che le decisioni vengano prese da 1-2 persone soltanto, ma incentivare la cooperazione, così che tutti siano più partecipi e aumenti il confronto di idee diverse; se qualcuno non collabora nella scelta delle azioni e nella disposizione dei post-it perché non ha capito qualcosa, spronarlo a fare domande
\item può essere necessario chiarire bene che la stessa azione può essere usata più volte, e in punti diversi dello stesso programma
\item alcuni gruppi utilizzano un approccio del tipo ``una certa azione viene eseguita/ripetuta finché non viene letta l’azione successiva”, in questo caso girare di spalle il portavoce in modo che non veda il robot durante la lettura delle azioni
\end{enumerate}

Nella seconda fase della attività a ciascun gruppo vengono distribuiti un foglio di carta, una penna e dei post-it di 4 colori diversi.

Dopo la prima fase di esplorazione vengono una serie di vincoli. Agli studenti viene chiesto di scrivere le indicazioni per il robot sul foglio di carta usando i post-it colorati. Su ciascun post-it è possibile scrivere un’azione da far compiere al robot, in modo sintetico, con 5 parole al massimo (ad esempio ``fai un passo in avanti” o ``rigarti a destra di 90 gradi”). Una volta associata un’azione ad un colore, tutti  i post-it di quel colore potranno essere usati solo
per quell’azione.

Gli studenti dovranno posizionare i post-it sul foglio per descrivere la sequenza di azioni da far compiere al robot.

Agli studenti viene data anche la possibilità di usare delle strutture di controllo (dei superpoteri), la cui sintassi è la seguente:
\begin{enumerate}
\item SE <condizione> ALLORA ESEGUI <post-it/azione>
\item RIPETI <post-it/azione> PER <n> VOLTE
\item FINO A QUANDO <condizione> RIPETI <post-it/azione>
\end{enumerate}

Gli studenti possono utilizzare le strutture di controllo inserendole nella sequenza di azioni sul foglio di carta e posizionando i post-it/azioni per scegliere quale azione eseguire o ripetere.


\textbf{prestare attenzione a / suggerimenti}
\begin{enumerate}
\item assicurarsi che tutti i gruppi scrivano i loro programmi e provino a testarli, quando la maggior parte delle coppie ha testato il proprio programma almeno una volta e verificato che più o meno funziona, si può passare alla fase di discussione
\item se un gruppo ha già definito una soluzione che “funziona” ma altri gruppi stanno ancora lavorando, si può cercare di stimolare il gruppo ad analizzare/migliorare il loro programma, ad esempio far fare il robot a un altro componente del gruppo, far girare il portavoce di schiena, spostare l’oggetto da raccogliere, eccetera...
\end{enumerate}

A conclusione dell’attività, a classe intera, si fa eseguire il programma a ciascun gruppo mentre gli altri gruppi osservano. Alla fine di ciascuna esecuzione, si chiede agli altri studenti se hanno osservazioni da fare. Se c’è tempo si possono fare altre ripetizioni scambiando robot e/o portavoce (scegliendo le combinazioni che mettono in evidenza le eventuali ambiguità o fragilità dei programmi).

Alla fine si riassumono e si mettono in evidenza le caratteristiche delle varie soluzioni:
\begin{enumerate}
\item contare i passi oppure uso del bordo del tavolo con la struttura di controllo “ripeti fino a quando”
\item soluzioni più o meno generali/dipendenti dal robot
\item scelta delle primitive e loro combinazione,  ad esempio ruota a destra di 90 gradi ripetuto 3 volte per girare anche a sinistra
\end{enumerate}

L’obiettivo di questa fase è far ragionare la classe sugli approcci proposti, partendo da un confronto tra le soluzioni proposte. È un passaggio delicato, cui dedicare un po’ di tempo, perché consente ai ragazzi di riflettere su quanto hanno fatto fino adesso.

\textbf{prestare attenzione a / suggerimenti}
\begin{enumerate}
\item cercare di fare riferimento il più possibile a esempi concreti che si riscontrano nelle proposte dei vari gruppi, incentivare le osservazioni da parte dei componenti degli altri gruppi
\item è probabile che in alcuni casi gli studenti contestino la correttezza delle soluzioni degli altri gruppi (il robot imbroglia, il portavoce modifica la lettura in base a quello che succede al robot, eccetera…), questa è una buona occasione per sottolineare il fatto che è necessario definire delle istruzioni precise e non ambigue
\item fare attenzione che l’aspetto competitivo non prevalga sul concetto che si vuole far emergere
\end{enumerate}

\section{Attività al computer}

Quello che i ragazzi hanno sperimentato è strettamente connesso al modo in cui il computer gestisce le informazioni. Infatti, il computer procede eseguendo istruzioni che è in grado di comprendere.
Per permettere agli studenti di poter sperimentare in prima persona la programmazione si introduce l’ambiente di programmazione ``scratch”. Nello specifico saranno proposti ai ragazzi degli esercizi che consistono proprio nel far uscire un “robot” (un’entità in grado di capire le istruzioni messe a disposizione da scratch) da un labirinto. Gli esercizi sono organizzati in ordine di complessità crescente e introducono anche gli aspetti formali di tematiche quali selezione e iterazione. 

Si procede alla divisione in coppie degli studenti. È importante che ogni coppia abbia accesso a un solo computer per incentivare il lavoro cooperativo. 

Quando le coppie si sono posizionate, introdurre a grandi linee l’ambiente di programmazione, usando il labirinto 0, senza suggerire possibili soluzioni:
\begin{enumerate}
\item Introdurre l’istruzione ``fai n passi" (mostrando come cambiare il numero di passi) e un’istruzione di rotazione, facendo vedere come eseguirle in sequenza per muovere la lampada in un percorso che non abbia particolare senso per il labirinto.
\item Introdurre il costrutto ``se - allora" spostando manualmente la lampada vicino all'uscita e facendo vedere come si verifica se il naso è su un colore piuttosto che no. Usare il fumetto come istruzione da eseguire in modo condizionale. Gli altri due costrutti (``ripeti n volte" e ``ripeti fino a quando") si menzionano solo per distinguerli dagli altri.
\item Indicare dove si trovano in generale i blocchi corrispondenti alle azioni, ai superpoteri e ai sensori.
\item Spiegare come caricare i file con i vari labirinti Quando tutti i gruppi hanno capito come lavorare, lanciare la gara: per contare la lunghezza di un programma valgono solo i mattoncini blu (non le strutture di controllo). Questo spingerà i ragazzi a usare i cicli quando possibili piuttosto che inserire istruzioni sequenziali ripetitive.
\end{enumerate}

Si consiglia di prestare attenzione alle seguenti situazioni:
\begin{enumerate}
\item A volte i ragazzi cercano di attraversare il percorso a pezzi (dispongono dei blocchi e li eseguono, poi li modificano e li eseguono, e così via fino all’uscita). In questo caso bisogna chiarire che l’obiettivo è invece scrivere un programma intero che, una volta avviato, porti l’oggetto dalla partenza all’uscita.
\item Se è il caso (ad esempio quando si vedono programmi molto lunghi), fare osservare la struttura ripetitiva dei labirinti e proporre di sfruttarla per ridurre il numero delle istruzioni attraverso l’uso dei costrutti di controllo se/ripeti.
\item Evitare che i ragazzi perdano tempo modificando gli sprite o usando altre funzionalità di scratch non pertinenti con gli obiettivi del laboratorio.
\end{enumerate}

\section{Conclusione}

In questa fase si ricapitola quanto si è fatto durante le varie attività per dare la possibilità alla classe di riflettere sul senso complessivo del laboratorio: ragionare sul concetto di informazione nel contesto dell’informatica come scienza che si occupa dell’elaborazione automatica dell’informazione, e in particolare ragionare su come si possa rappresentare l’informazione relativa alla formattazione dei testi in maniera tale da rendere automatica la loro elaborazione (nelle pagine web, tramite un programma di videoscrittura, ecc).

Anche questa fase può essere condotta o in modo interlocutorio, o in maniera frontale. Nel primo caso le domande possono essere del tipo:
\begin{enumerate}
\item siamo partiti parlando della parola ``automatico": che cosa c’è di automatico in quello che abbiamo fatto?
\item si può dire che abbiamo fatto ``informatica" oggi? quando? anche nella attività con i robot umani?
\end{enumerate}

Ecco gli elementi che il conduttore deve cercare di fare emergere oppure proporre nel suo riassunto:
\textit{Automatica}. Il robot (anche se umano!) doveva eseguire le istruzioni senza ``metterci del suo", addirittura senza percepire elementi dell’ambiente che non fossero stati esplicitamente menzionati dai ``programmatori", cioè i progettisti delle attività del robot. Invece non c'è niente di automatico nel progettare e scrivere un programma che eseguito da ``esecutori automatici" consenta di ottenere l’obiettivo desiderato (c'è bisogno infatti di logica, inventiva, creatività).
\textit{Informazione}. Sono in gioco almeno due tipi molto diversi di informazioni: da un lato c'è il programma, che deve essere descritto in termini ``formali" (cioè rispettando regole di forma, come l’uso dei post-it) con l'obiettivo di eliminare le ambiguità; dall’altro ci sono i dati sull’ambiente in cui si trova il robot che vengono rappresentati (e tipicamente semplificati/stilizzati) nel programma: per esempio, il fatto che nella stanza ci potesse essere un profumo, non entrava a far parte delle cose percepite dal robot (anche se la persona robot, lo avrebbe certamente percepito).
\textit{Elaborazione}. L’informazione rappresentata viene manipolata, elaborata e trasformata in movimenti del robot.

Se si riesce, anche in questa fase è utile cercare di valorizzare quanto emerge dalla classe. Ad esempio, se qualcuno si dichiara poco convinto di aver fatto informatica, cogliete l’occasione per cercare di spiegare meglio in che senso invece stabilire che istruzioni dare a un robot umano è informatica (analizzare problemi e studiare soluzioni che consentono poi una elaborazione automatica dell’informazione: sfruttare il bordo del tavolo per decidere quando girare è una buona idea ``informatica", istruzioni che dipendono fortemente dalla lunghezza del piede del robot umano invece non lo sono dato che forniscono la soluzione solo in alcuni casi particolari).

\end{document}

